\nomenclature{ABI}{Application Binary Interface: The interface
	between binaries.  ABI-compatible binaries
	can interface with each other.}
\nomenclature{ASIC}{Application-Specific Integrated Circuit (or Cloud ;-)}
\nomenclature{CAS}{Compare-And-Swap; atomic operation}
\nomenclature{CPU}{Central Processing Unit; in this document the
	term is context-dependant.  It is used to denote both the
	physical hardware unit or a virtual concept of a CPU.}
\nomenclature{DMA}{Direct Memory Access}
\nomenclature{DSL}{Domain Specific Language}
\nomenclature{DSO}{Dynamic Shared Object}
\nomenclature{ELF}{Executable and Linking Format; the binary format
	used by NetBSD and some other modern Unix-style operating
	systems.}
\nomenclature{FFS}{Berkeley Fast File System; in most contexts,
	this can generally be understood to mean the same as UFS
	(Unix File System).}
\nomenclature{FS}{File System}
\nomenclature{GPL}{General Public License; a software license}
\nomenclature{HTTP}{HyperText Transfer Protocol}
\nomenclature{i386}{Intel 32-bit ISA (a.k.a. IA-32)}
\nomenclature{IPC}{Inter-Process Communication}
\nomenclature{ISA}{Instruction Set Architecture}
\nomenclature{LRU}{Least Recently Used}
\nomenclature{LGPL}{Lesser GPL; a less restrictive variant of GPL}
\nomenclature{LWP}{Light Weight Process; the kernel's idea of a thread.
	This acronym is usually written in lowercase (lwp) to mimic the
	kernel structure name (\texttt{struct lwp}).}
\nomenclature{MD}{Machine Dependent [code]; [code] specific to the
	platform}
\nomenclature{MI}{Machine Independent [code]; [code] usable on all
	platforms}
\nomenclature{MMU}{Memory Management Unit: hardware unit which handles
	memory access and does virtual-to-physical translation for
	memory addresses}
\nomenclature{NIC}{Network Interface Controller}
\nomenclature{OS}{Operating System}
\nomenclature{OS}{Orchestrating System}
\nomenclature{PCI}{Peripheral Component Interconnect; hardware bus}
\nomenclature{PIC}{Position Independent Code; code which uses relative
	addressing and can be loaded at any location.  It is typically
	used in shared libraries, where the load address of the code
	can vary from one process to another.}
\nomenclature{PR}{Problem Report}
\nomenclature{RTT}{RoundTrip Time}
\nomenclature{RUMP}{Deprecated ``backronym'' denoting a rump kernel
	and its local client application.  This backronym should
	not appear in any material written since mid-2010.}
\nomenclature{SLIP}{Serial Line IP: protocol for framing IP datagrams
	over a serial line.}
\nomenclature{TLS}{Thread-Local Storage; private per-thread data}
\nomenclature{TLS}{Transport Layer Security}
\nomenclature{UML}{User Mode Linux}
\nomenclature{USB}{Universal Serial Bus}
\nomenclature{VAS}{Virtual Address Space}
\nomenclature{VM}{Virtual Memory; the abbreviation is context dependent and
	can mean both virtual memory and the kernel's virtual memory
	subsystem}
\nomenclature{VMM}{Virtual Machine Monitor}
